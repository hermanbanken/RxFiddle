\documentclass[11pt,a4paper]{article}
\usepackage{graphicx}
\usepackage{url}
\usepackage[top=1.25in, bottom=1.25in, left=1.25in, right=1.25in]{geometry}

\begin{document}

\title{
  Thesis Proposal \\
  Visualizing Data-Flows in Reactive Programs
}
\author{
  Herman Banken - 4078624 \vspace{1cm} \\
  TU Delft, Department of Electrical Engineering, \\
  Mathematics and Computer Science
}
\date{\today}

\vfill

\begin{figure}[!bp]
  \centering
  \begin{minipage}[b]{0.4\textwidth}
    \includegraphics[width=\textwidth,trim={0 12cm 0 10cm 0},clip]{images/tudelft.pdf}
  \end{minipage}
  \hfill
  \begin{minipage}[b]{0.4\textwidth}
    \includegraphics[width=\textwidth]{images/ordina.png}
  \end{minipage}
\end{figure}
	
\maketitle

\clearpage

\section{Introduction}
Reactive Programming has been around for many years, and recently several implementations have surfaced that are now incorporated into widely used frameworks and  in use for many production applications. However, adapting to the paradigm of Reactive Programming as a developer is known to take some time. Furthermore, even with basic or advanced knowledge of an implementation it takes longer to understand existing Reactive code than simple sequential code. What can really help to understand the created data flows are diagrams~\cite{weck2016visualizing} which show how data is received, transformed and emitted at each step.

The scope of this thesis will be Reactive Extensions (Rx)~\cite{msdn_rx}, one of the libraries implementing Reactive Programming which has implementations in almost every programming language. The main documentation of Rx~\cite{reactivex} uses Marble Diagrams~\cite{c9_marblediagrams} for each operator to show the behaviour of this single operator. These diagrams really complement the name of the operator and its description, allowing the developer to work-out the nitty details and pick the right operator for it's use. They are however only generated per operator, and are not combined for complete data flows, showing the full flow through many operators.

In this master's thesis we will focus on creating complete and interactive Marble Diagrams for full data flows, automatically, from sources and from running applications.

\subsection{Observable structure analysis}
The template for the data flows, encapsulated in Observable in Rx, are contained in code. By analysing the source code or bytecode these templates can be extracted. Observables are created by calling several factory methods on the Observable-class. After creation they can be passed as variables and can be transformed by applying operators which generate a new, extended Observable structure. Since Observables are (immutable) value types they can be used multiple times as a basis to create new structures, therefore possibly creating a tree of related Observable structures. This structure is the basis for the visualisation.

\subsection{Run-time analysis}
By analysing the structure we already know through which operators possible future data will flow. During run-time we can detect this propagation of data through operators. In Rx the methods onNext, onError and onComplete propagate data, which can be instrumented to log the invocation to the visualisation engine. Every event then gets shown as a marble in the correct Observable axis.

AspectJ: AOP

\subsection{Generating data}
Testing tools like QuickCheck~\cite{quickcheck} automate test generation by producing arbitrary input, and by finding test cases that falsify the test conditions. When a falsification is found QuickCheck tries to simplify the test data, pruning data which does not attribute to the tests failure. An equally advanced test tool for data flows would be interesting, but is not in scope. Generating data however can be interesting. Visualising the behaviour of Observables without running the actual program, based solely on the data flow structure and generated data could provide valuable insight. Learning from QuickCheck, reducing to pivotal data can show the various edge cases of how a data flow can evaluate while keeping the amount of cases to be considered (and interpreted by humans) at a minimum.

\subsection{Tainting}
When looking at the values bubbling through an Observable structure, values might be produced which are not directly relatable to their sources. With pointwise transformations the developer can trace each output back to a single point of input. However, operations that fold over time might both use new and reuse older values. The relation between these variables might not be clear over time. One existing solution to track dependencies between variables is called tainting~\cite{bell2015dynamic}: by applying a taint to a variable, dependent variables either get the same taint or a mixture of all the taints of it's dependencies. Implementations of tainting like Phosphor~\cite{bell2014phosphor} can be evaluated and might be interesting to integrate.

\section{Research Questions}
The main research question is:

\begin{quotation}
	\noindent
	Does visualising Reactive Extensions help developers comprehend their code and ease the debugging of Observables?
\end{quotation}

\noindent 
Several smaller questions must be answered to answer the main question:

\begin{enumerate}
	
\item Structures:
\begin{enumerate}
	\item Can Observables structures be represented in an abstract fashion?
	\item Can we extract Observable structures from source code or bytecode?
	\item Can we extract run-time behaviour of Observables such that it is appropriate input to a visualizer / simulation?
	\item Can we generate smart test input data for Observable structures?
\end{enumerate}

\item Visualisation:
\begin{enumerate}
	\item Can Marble Diagrams effectively convey structures containing more than 1 operator?
\end{enumerate}

\item Debugger Usability:
\begin{enumerate}
	\item Can our tool (fully) replace traditional print-debugging in practice?
	\item Do developers use automated test data in practice?
	\item Does our tool improve the development experience when working with Rx?
\end{enumerate}

\end{enumerate}


\section{Motivation}
Reactive Programming is a different programming model than sequential programming. Traditional development tools are not completely adequate for this model: just like Async Programming now has async call stacks in Chrome, Reactive Programming also needs more tools to make developing with the technology easier and less painful. 

Anecdotal evidence and personal experience suggest developers tend to just add print and debug statements through the reactive code to get a sense of the ordering and effects of events over time. This style of debugging requires constantly changing the source code, adding and removing print statements on the go. When using dynamic languages this can be a little annoying, but when using statically typed languages the recompilation can take long and the process slows down the development.

Another approach - which might be preferable to print-debugging - is creating tests. However, in large Observable structures there might be many variation points which would require exponentially many variations of (minimal) input events to be tested or even to be considered. After a bug occurs it might be impossible to know in which state the full Observable was. Therefore exploring the different scenarios in a visual way might help to reproduce the bug. Especially when giving instant feedback on how a small change in input changes the output over time, by using virtual time schedulers. 

Sidenote: the above paragraph also presents another option: keeping track of the state (in a smart \& efficient way) and being able to dump this when something unexpected happens. One problem with this solution is that something unexpected can also be the absence of an occurrence, which is not necessarily detectable.

A reason for scoping this to Rx is how wide Rx is used but mainly how mature it is. The implementation of Rx dates back before 2010 and is very well thought through while newer frameworks like Reactive Streams and Bacon.js lack these backgrounds. Rx is very stable and structured, simplifying the implementation of the prototype. When completed, it can then be easily extended to many other Reactive Programming implementations.

\section{Planning}

\subsection{Scheme}

What
Deliverable
When
Start at Ordina
-
12th of September, 2016
Thesis Proposal
Thesis Proposal
25th of September
Test prototype
Prototype implementation
1st of December
Thesis Draft
Draft of final report
15th of March, 2017
Thesis Defense
Presentation
15th of April

\subsection{Risk analysis}
(= still remains to be done)

Openstaande vak IN4306 Literatuurstudie
Te grote scope
Meijer's tijd

\subsection{Contact}

\begin{table}[h]
\centering
\begin{tabular}{@{}lll@{}}
\textbf{Student}       & \textbf{University}     & \textbf{Ordina}        \\\hline
Herman Banken          & Prof.dr. H.J.M. Meijer  & Joost de Vries         \\
B.vd.Polweg 498, Delft & EWI HB08.060 / SV       & Ringwade 1, Nieuwegein \\
06 - 38 94 37 30       & -                       & 06 - 12 89 56 76       \\ 
hermanbanken@gmail.com & H.J.M.Meijer@tudelft.nl & Joost.de.Vries@ordina.nl
\end{tabular}
\end{table}

\section{Supervision details}
(= still in draft-state)

Weekly meetings with the company supervisor Joost de Vries
Bi-Weekly meetings with Erik Meijer over Skype

Ordina provides a working place, computer for the thesis, as well as sparring partners in the form of other students and colleagues of Joost from Code Star and SMART on the same floor.

Furthermore some 'user' (developer) tests will need to be executed, 
to learn existing workflows;
to compare existing workflows to new proposed workflows;
to provide input on the usability of the tools;
or to measure satisfaction with the new tools
for which it would also be very convenient if some employees of Ordina could volunteer.

\bibliography{papers/references}{}
\bibliographystyle{plain}

%Visualizing Data-Flows in Functional Programs
%http://doi.ieeecomputersociety.org/10.1109/SANER.2016.82
%
%Reactive Extensions, Microsoft
%https://msdn.microsoft.com/en-us/data/gg577609.aspx
%
%Platform independent Reactive Extensions documentation
%https://reactivex.io
%
%Marble Diagrams explained in Channel9 video, Microsoft
%https://channel9.msdn.com/blogs/j.van.gogh/reactive-extensions-api-in-depth-marble-diagrams-select--where
%
%QuickCheck
%http://www.cs.tufts.edu/~nr/cs257/archive/john-hughes/quick.pdf
%
%Dynamic Taint Tracking for Java with Phosphor
%https://mice.cs.columbia.edu/getTechreport.php?techreportID=1601
%
%Phosphor on Github
%https://github.com/Programming-Systems-Lab/phosphor


\end{document}
