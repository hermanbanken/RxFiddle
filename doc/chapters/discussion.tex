\section{Discussion}
\subsection{Usability}
Experienced RP developers get more speedup from RxFiddle than inexperienced developers. This indicates that the debugger design might require more training than currently supplied. The visualization is not as easy to understand as we initially hoped. A possible explanation for the differences in the results of the larger tasks T3 and T4, is the visualization. T4 requires switching between merged Observables to gain a full view of the events, while the root cause of the error in T3 is directly visible. If we compare our visualization to the visualization of other tools like the soon to be released visualizer of Misha Moroshko\footnote{\url{https://twitter.com/moroshko/status/854529439262167040}}, the main difference is that RxFiddle shows a linear Marble Diagram view of a single flow or path through the graph, while other visualizers show a `multi-path diagram'. The choice to show a single path allows us to show long paths, but is restricting as showing multiple paths would create complicated visualizations which we deemed too cluttered to be usable. For future research it would be very interesting to compare the different ways Observable streams can be combined in Marble Diagrams and which visualization elements can be added to explicitly show the causality and lineage of events and show durations of subscriptions.

\subsection{Visualization}
Our implementation visualizes higher order flows as a combination of the original flow with incoming edges from the merged flows. While this visualization matches the behavior of merging streams together it does not map conceptually with some other use cases of higher order flows, for example using one Observable as stop condition for another Observable (\code{takeUntil}).
