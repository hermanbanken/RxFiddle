\section{Background}
In this section we give an overview of related work 
and the context of this research.

\textbf{Debugging for Program Comprehension.}
Both debugging and comprehension are processes in the work of programmers.
Initially comprehension was seen as a distinct step programmers had to make
prior to being able to debug programs~\cite{katz1987debugging}, 
but this distinction is criticized by Gilmore saying we must view 
``debugging as a design activity''~\cite{gilmore1991models}, 
part of creating and comprehending programs. 
Maalej et al.~\cite{Maalej2014} interviewed professional developers 
and found that developers require runtime information to understand a program,
and that debugging is frequently used to gather this runtime information.
This supports our view that `debugging' is not only used for fault localisation,
but also for comprehension.

\textbf{Dynamic Analysis.}
Although interactive debuggers are commonly used for comprehension 
of the runtime behavior of programs, more specialized tools exist: 
to study a programs execution is called `dynamic analysis' which has 
received substantial attention in the research community,
as surveyed by Cornellissen et al.~\cite{cornelissen2009systematic}.
They categorise on different facets being the 
\textit{activity} [goal of analysis],
\textit{target} [kind of inspected program or system],
\textit{method} [visualization, metrics, online, querying, etc.],
\textit{evaluation} [preliminary, case study, quantitative, etc.].
Most papers apply a post mortem analysis were the program is run,
and then analyse the trace data to create a visualization. 
Reis et al. mention the compromises
that have to be made to make an online analysis: 
reduced tracing is required to not slow down the 
system (known as the observer-effect), fast analysis 
and visualization is required to lower the cost of getting 
to the visualization, to not discourage the users.

Only X of the surveyed papers apply post mortem analysis.

\textbf{Measuring Debugging.}

\textbf{Debugging for specific fields}
Most research into debugging focusses on procedural and 
imperative languages~\cite{cornelissen2009systematic}.
Other topics of interest are multi-threading and distributed systems.
and only few focus on other styles, like 
declarative programming~\cite{nilsson1998declarative}.
Debugging in reactive programming~\cite{
	salvaneschi2014empirical,salvaneschi2016debugging}.
